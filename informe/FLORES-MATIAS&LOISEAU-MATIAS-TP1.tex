

% This is a semi-simple sample document.

\documentclass{article} % \documentclass{} is the first command in any LaTeX code.  It is used to define what kind of document you are creating such as an article or a book, and begins the document preamble

%\usepackage[spanish]{babel} % Para traducir palabras claves al español como "Figure" -> "Figura"

\usepackage{amsmath, amssymb} % \usepackage is a command that allows you to add functionality to your LaTeX code

\usepackage{mathtools} % using cases inside equations

\usepackage{graphicx} % image package

\usepackage{float} % Float specifiers are written in the square brackets whenever we use a float such as a figure or a table

\usepackage{subcaption} % more than one picture s in the same figure

\parskip 0.1in %paragraph distance

\usepackage[margin=0.984252in]{geometry} %margin

\usepackage[hidelinks]{hyperref} % Magic for index linking

\usepackage{chngcntr} % For figure number matching with section number
\counterwithin{figure}{section}

\usepackage{appendix} % Appendix

\title{Trabajo Práctico N°1} % Sets article title
\date{Septiembre 2023} % Sets article date

\author{
	\textbf{Fundamentos de Análisis de Datos}\\
	Maestría en Ciencia de Datos\\
	\\~\\
	\textbf{Flores, Matías}\\
	matflores@itba.edu.ar
 	\\~\\
 	\textbf{Loiseau, Matías}\\
 	mloiseau@itba.edu.ar
}

% The preamble ends with the command \begin{document}
\begin{document} % All begin commands must be paired with an end command somewhere

\begin{figure}
\centering
	\includegraphics[width=0.2\textwidth]{images/itba-logo}
	%\caption{Mi Figure}
	\label{fig:itba-logo}
\end{figure}
\maketitle % creates title using information in preamble (title, author, date)

\thispagestyle{empty} % Ignore page number
\cleardoublepage

\cleardoublepage
\tableofcontents % general index
\cleardoublepage

\section{Ejercicio N° 1}
En el archivo \textit{Dieta.xlsx} se encuentran los datos correspondientes a 173 personas que están siguiendo una dieta. Para cada una de ellas, se registró el sexo y el consumo de grasas saturadas y de alcohol, así como del total de calorías diarias.

\subsection{Primer punto}

\textbf{Consigna:} Analizar si existen datos faltantes y, en caso afirmativo, eliminar tales registros.



\subsection{Segundo punto}

\textbf{Consigna:} Calcular las siguientes medidas estadísticas descriptivas clásicas del consumo de grasas: rango, media, mediana, desvío estándar y rango intercuartil.


\subsection{Tercer punto}

\textbf{Consigna:} Realizar gráficos boxplots de los datos sobre el consumo de calorías en función de la variable categórica. ¿Qué puede observarse?


\subsection{Cuarto punto}

\textbf{Consigna:} Dividir la cantidad de calorías consumidas en dos categorías: MODERADA (menor o igual a 1700) o ALTA (mayor a 1700). Analizar el consumo de alcohol de acuerdo a la cantidad de calorías consumidas según las categorías definidas.

\section{Ejercicio N° 2}
El archivo \textit{Sociodemograficos.xlsx} contiene datos sobre distintos indicadores socio-demográficos de varios países.

\subsection{Primer punto}

\textbf{Consigna:} ¿Cuáles son las variables de interés? ¿Cuántos países fueron analizados?

\subsection{Segundo punto}

\textbf{Consigna:} ¿Cuáles son los países con menor y mayor tasa de natalidad?


\subsection{Tercer punto}

\textbf{Consigna:} Realizar un diagrama de dispersión con las tasas de natalidad y de mortalidad infantil. ¿Qué puede observarse? Justificar lo observado a partir del gráfico con una medida cuantitativa.


\subsection{Cuarto punto}

\textbf{Consigna:} Calcular el vector de medias y medianas.

\subsection{Quinto punto}

\textbf{Consigna:} Calcular las matrices de covarianzas y de correlaciones. A partir de estas matrices dar un ejemplo de dos variables fuertemente correlacionadas positivamente, de dos variables fuertemente correlacionadas negativamente y de dos variables no correlacionadas.

\section{Ejercicio N° 3}
Vamos a considerar el conjunto de datos \textit{swiss} disponible en R.

\subsection{Primer punto}

\textbf{Consigna:} Cargar la base de datos y explorarla. ¿Cuántos registros y cuántas variables tiene? Describir las variables de estudio.

\subsection{Segundo punto}

\textbf{Consigna:} Se desea comparar las provincias entre sí. ¿Es adecuado utilizar la distancia Euclídea para realizar la comparación? Justificar la respuesta.


\subsection{Tercer punto}

\textbf{Consigna:} Buscar la presencia de datos atípicos mediante la distancia de Mahalanobis. Comentar los resultados obtenidos.


\section{Ejercicio N° 4}
El Departamento de Psicología de una universidad ubicada en una ciudad céntrica realizó un estudio sobre la asistencia a clases teóricas no obligatorias dependiendo de la localidad de residencia del estudiantado. Para tal fin, se seleccionaron 40 estudiantes en la Ciudad A, 40 estudiantes en la Ciudad B y 40 estudiantes en la Ciudad C, y se contabilizó la cantidad de clases a las que cada uno/a asistió. Los resultados obtenidos se muestran en la siguiente tabla.

\begin{table}[H]
	\centering
		\begin{tabular}{|| c | c | c ||}
			\hline
			\hline
			Ciudad A & Ciudad B & Ciudad C\\
			\hline
			\hline		
			11 & 13 & 6\\
			\hline
			14 & 10 & 7\\
			\hline
			7 & 12 & 3\\
			\hline
			15 & 7 & 5\\
			\hline
			11 & 5 & 9\\
			\hline
			13 & 10 & 6\\
			\hline
			11 & 10 & 1\\
			\hline
			16 & 16 & 6\\
			\hline
			10 & 9 & 0\\
			\hline
			15 & 7 & 2\\
			\hline
			18 & 7 & 5\\
			\hline
			12 & 2 & 6\\
			\hline
			9 & 6 & 11\\
			\hline
			9 & 9 & 6\\
			\hline
			10 & 9 & 7\\
			\hline
			10 & 8 & 0\\
			\hline
			15 & 8 & 5\\
			\hline
			10 & 10 & 7\\
			\hline
			14 & 3 & 5\\
            \hline
            10 & 6 & 4\\
            \hline
            10 & 5 & 7\\
            \hline
            12 & 2 & 4\\
            \hline
            14 & 9 & 2\\
            \hline
            12 & 3 & 8\\
            \hline
            15 & 4 & 9\\
            \hline
            7 & 5 & 6\\
            \hline
            13 & 10 & 1\\
            \hline
            6 & 8 & 4\\
            \hline
            10 & 5 & 7\\
            \hline
            15 & 9 & 7\\
            \hline
            20 & 10 & 8\\
            \hline
            10 & 8 & 9\\
            \hline
            13 & 13 & 7\\
            \hline
            10 & 10 & 5\\
            \hline
            6 & 0 & 1\\
            \hline
            14 & 2 & 6\\
            \hline
            8 & 1 & 9\\
            \hline
            10 & 1 & 4\\
            \hline
            8 & 0 & 7\\
            \hline
            11 & 4 & 16\\
            \hline
			\hline
		\end{tabular}
		\caption{Table de asistencia a clases teóricas no obligatorias.}
	\label{tab:table-x}
\end{table}

\subsection{Primer punto}

\textbf{Consigna:} Armar un \textit{data frame} en \textit{R} con los datos de la tabla anterior, creando dos variables: una que represente la cantidad de asistencias a las clases teóricas no obligatorias y otra que represente la localidad de residencia. ¿Qué tipo de variable es cada una?

\subsection{Segundo punto}

\textbf{Consigna:} Analizar los datos de la muestra mediante gráficos y medidas estadísticas descriptivas. ¿Se observan diferencias en los valores promedios por localidad?

\subsection{Tercer punto}

\textbf{Consigna:} Realizar un test ANOVA para comparar las medias de las 3 poblaciones. Plantear las hipótesis nula y alternativa del test, informar los resultados obtenidos y la decisión tomada.


\subsection{Cuarto punto}

\textbf{Consigna:} Si se han obtenido diferencias significativas entre las localidades, determinar cuáles son esas diferencias utilizando el test de Tukey.





\cleardoublepage

\begin{thebibliography}{9}
\addcontentsline{toc}{section}{Bibliography} % This is for the bibliography appears in the index

\bibitem{iot}
Iván Federico Kwist, Matías Loiseau, David Exequiel Contreras, Federico Gabriel D’Angiolo,  Roberto Osvaldo Mayer. (2019). \textit{Monitorización de un Datacenter mediante Protocolos de IoT}. Congreso Nacional de Ingeniería Informática – Sistemas de Información.

\bibitem{regresion-lineal}
Federico Gabriel D’Angiolo, Iván Federico Kwist, Matías Loiseau, David Exequiel Contreras, Fernando Asteasuain. (2019). \textit{Algoritmos de Regresión Lineal aplicados al mantenimiento de un Datacenter}. Congreso Argentino de Ciencias de la Computación.

\bibitem{knn}
Federico Gabriel D’Angiolo, Iván Federico Kwist, Matías Loiseau , David Exequiel Contreras, Gregorio Oscar Glas. (2019). \textit{Algoritmo de KNN aplicado al mantenimiento de un Datacenter}. Congreso Nacional de Ingeniería Informática – Sistemas de Información.

\bibitem{deep-learning-nature}
LeCun, Y., Bengio, Y., \& Hinton, G. (2015). \textit{Deep learning}. nature, 521(7553), 436-444.

\bibitem{object-detection-review}
Zhao, Z. Q., Zheng, P., Xu, S. T., \& Wu, X. (2019). \textit{Object detection with deep learning: A review}. IEEE transactions on neural networks and learning systems, 30(11), 3212-3232.

\end{thebibliography}


\end{document} % This is the end of the document
